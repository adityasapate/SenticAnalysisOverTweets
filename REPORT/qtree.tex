%
% qtree.sty, a LaTeX macro package for drawing trees from descriptions
% given in bracket notation.
%
    \def\qTreeVersion{3.1b}
    \def\qTreeDate{2008/12/12 }

% QobiTree tree macros written by Jeffrey Mark Siskind (Qobi@cis.upenn.edu).
% Front end and extensions by Alexis Dimitriadis (alexis@ling.upenn.edu).
%
% Please direct questions/comments to alexis@ling.upenn.edu.
% 

% Most recent revisions:

% 2008/12/12 -- 3.1b A roof with an empty label will be connected to the tree
%               branches (but will look ugly because of the different angles)

% 2006/03/08 -- Protected pict2e loading from possible redefinition of ^.
%               Reported by Alan Munn. Extended documentation.

% 2005/12/22 -- 3.0b Removed pgf mode, switched to pict2e.sty support for
%               everything. Updated documentation.

% 2005/05/22 -- Added customization hooks and `balanced' branch command,
%               fixed final whitespace in \leaf. Many small tweaks. 

% 2005/05/13 -- Reorganized mode selection options, merged variant file 
%               forks.

% 2005/04/07 -- Removed errant spaces, and arranged for trees to be truly
%               left-aligned or centered.

% 2005/02/13 -- Added pgf drawing mode, mode selection support.

% 2004/07/05 -- Added a way to manipulate the height of single-child nodes.

% 2002/11/05 -- Fixed display bug in first branch of 5-branching nodes.
%               Reported by Paliath Narendran [dran@cs.albany.edu]. 

% 2002/04/17 -- Added provision for ``zero branching'' (empty non-terminals),
%               as suggested by Ronald Greenberg [rig@cs.luc.edu].

% 2001/09/01 -- Finally made LaTeX2e aware; eepic inclusion is now
%               controllable by a package option; massively revised
%               documentation.

% 2000/10/24 -- Fixed bug in last revision that broke parsing of \qroof. 

% 2000/9/28  -- Fixed handling of empty labels so that tree lines stay
%               connected.
%            -- Small spacing fixes: changed \0, \1 and \2 to \rlap their
%               output so that it does not impact on centering.

\typeout{Package: \qTreeDate v.\qTreeVersion Qtree: tree-drawing for
  linguistics} 

% Attempt to preserve LaTeX 2.19 compatibility by not evaluating package
% option commands unless they are defined. Probably pointless...
%  
\ifx\ProvidesPackage\undefined  \else
  \ProvidesPackage {qtree} [\qTreeDate v.\qTreeVersion Qtree: tree-drawing for
  linguistics] 

  \newif\ifqtreecenter \qtreecentertrue
  \DeclareOption{nocenter}{\qtreecenterfalse}
  \DeclareOption{center}{\qtreecentertrue}
  
  % Control automatic loading of graphics extensions
  \newif\ifq@autoload \q@autoloadtrue
  \DeclareOption{noload}{\q@autoloadfalse}

  % Compatibility options
  \def\q@usage{Enhanced picture support for
both pdf and PostScript is now provided by the package pict2e.sty.
(If you need to suppress loading of pict2e.sty, use option [noload]).}

  \DeclareOption{pdf}{\typeout{WARNING: Option [pdf] is obsolete. \q@usage}}

  \DeclareOption{noeepic}{
    \typeout{WARNING: Option [(no)eepic] is obsolete. \q@usage}}
  
  \ProcessOptions

  % Autoload graphics extensions, unless [noload] was given.
  \ifq@autoload
    % Some linguistics styles like to redefine ^, which causes problems
    % to pict2e.sty. We protect it by temporarily forcing the default 
    % catcode for ^.
    \edef\svcatcode@up@{\the\catcode`\^}
    \catcode`\^=7
    	 % DON'T COMMENT THIS OUT! If you don't want pict2e,  	
    	 % just call qtree like this: \usepackage[noload]{qtree} 
       \RequirePackage{pict2e}
    % restore the catcode of ^ to whatever value it had
    \catcode`\^=\svcatcode@up@ \relax
    \let\sv@up@catcode=\relax
  \fi
  
\fi  % end of package options code


% Begin QobiTree code

\newcounter{treecount} \setcounter{treecount}{0}
\newcounter{branchcount}
\newsavebox{\parentbox}
\newsavebox{\treebox}
\newsavebox{\treeboxone}
\newsavebox{\treeboxtwo}
\newsavebox{\treeboxthree}
\newsavebox{\treeboxfour}
\newsavebox{\treeboxfive}
\newsavebox{\treeboxsix}
\newsavebox{\treeboxseven}
\newsavebox{\treeboxeight}
\newsavebox{\treeboxnine}
\newsavebox{\treeboxten}
\newsavebox{\treeboxeleven}
\newsavebox{\treeboxtwelve}
\newsavebox{\treeboxthirteen}
\newsavebox{\treeboxfourteen}
\newsavebox{\treeboxfifteen}
\newsavebox{\treeboxsixteen}
\newsavebox{\treeboxseventeen}
\newsavebox{\treeboxeighteen}
\newsavebox{\treeboxnineteen}
\newsavebox{\treeboxtwenty}
\newlength{\treeoffsetone}
\newlength{\treeoffsettwo}
\newlength{\treeoffsetthree}
\newlength{\treeoffsetfour}
\newlength{\treeoffsetfive}
\newlength{\treeoffsetsix}
\newlength{\treeoffsetseven}
\newlength{\treeoffseteight}
\newlength{\treeoffsetnine}
\newlength{\treeoffsetten}
\newlength{\treeoffseteleven}
\newlength{\treeoffsettwelve}
\newlength{\treeoffsetthirteen}
\newlength{\treeoffsetfourteen}
\newlength{\treeoffsetfifteen}
\newlength{\treeoffsetsixteen}
\newlength{\treeoffsetseventeen}
\newlength{\treeoffseteighteen}
\newlength{\treeoffsetnineteen}
\newlength{\treeoffsettwenty}

\newlength{\treeshiftone}
\newlength{\treeshifttwo}
\newlength{\treeshiftthree}
\newlength{\treeshiftfour}
\newlength{\treeshiftfive}
\newlength{\treeshiftsix}
\newlength{\treeshiftseven}
\newlength{\treeshifteight}
\newlength{\treeshiftnine}
\newlength{\treeshiftten}
\newlength{\treeshifteleven}
\newlength{\treeshifttwelve}
\newlength{\treeshiftthirteen}
\newlength{\treeshiftfourteen}
\newlength{\treeshiftfifteen}
\newlength{\treeshiftsixteen}
\newlength{\treeshiftseventeen}
\newlength{\treeshifteighteen}
\newlength{\treeshiftnineteen}
\newlength{\treeshifttwenty}
\newlength{\treewidthone}
\newlength{\treewidthtwo}
\newlength{\treewidththree}
\newlength{\treewidthfour}
\newlength{\treewidthfive}
\newlength{\treewidthsix}
\newlength{\treewidthseven}
\newlength{\treewidtheight}
\newlength{\treewidthnine}
\newlength{\treewidthten}
\newlength{\treewidtheleven}
\newlength{\treewidthtwelve}
\newlength{\treewidththirteen}
\newlength{\treewidthfourteen}
\newlength{\treewidthfifteen}
\newlength{\treewidthsixteen}
\newlength{\treewidthseventeen}
\newlength{\treewidtheighteen}
\newlength{\treewidthnineteen}
\newlength{\treewidthtwenty}
\newlength{\daughteroffsetone}
\newlength{\daughteroffsettwo}
\newlength{\daughteroffsetthree}
\newlength{\daughteroffsetfour}
\newlength{\branchwidthone}
\newlength{\branchwidthtwo}
\newlength{\branchwidththree}
\newlength{\branchwidthfour}
\newlength{\parentoffset}
\newlength{\treeoffset}
\newlength{\daughteroffset}
\newlength{\branchwidth}
\newlength{\parentwidth}
\newlength{\treewidth}
%

% Fine-tuning of trees

% The height of the line dominating 1-branching nodes. Can be redefined
% at any point to change behaviour (even in mid-tree).
\def\qtreeunaryht{2ex}

% Amount of padding inserted around leaves and labels
\def\qtreepadding{\tabcolsep}

% Amound of padding under the eaves of roofs.
\def\qroofpadding{0.4em}

% Typeset a leaf or label, wrapping with padding. Multiple lines are
% centered, unless the optional argument [c] is overridden (i.e., by \qroof)
% The first non-optional argument (#2) is usually \qleafhook or \qlabelhook.
% This command used to be called \ontop (also it used to be a lot simpler!)
\newcommand{\qshow@text}[3][c]{{#2{\begin{tabular}[t]
    {@{\hskip\qtreepadding}#1@{\hskip\qtreepadding}}#3\end{tabular}}}}

% A ``leaf'' may actually be a roof; if so, we pad it normally but suppress
% the hook; leaf and label hooks are applied separately to the parts of the
% roof. 
\def\qshow@leaf#1{\ifx\@qseenode\qroof \qshow@text{\relax}{#1}\else
  \qshow@text{\qleafhook}{#1}\fi}


\let\qtreeinithook=\relax  % global formatting/definitions hook
\let\qtreefinalhook=\relax % called just after the tree is output
\let\qleafhook=\relax     % hook for leaf text 
\let\qlabelhook=\relax    % hook for non-terminal labels
\let\qbranch@hook=\relax  % Called after completing a branch (undocumented)

% Don't take up space for empty labels; as a result, branch segments will be
% attached to their parent branch when no label is given.
% (\rlap triggers space in {tabular}, so it's NON zero, and will be displayed).
\def\showparentbox{\ifdim\wd\parentbox>0pt
     \hspace*{\parentoffset}\usebox{\parentbox}\\\fi}


% Adjust the bottom-most fork in a binary branching tree, so that branches
% will be of even size if the labels are.
% Use:
% Invoke just before closing the bottom-most (binary) fork; for example:
%    \Tree [.X label_1 [.Y label_2 [.Z label_3 right !\qbalance{} ]
% This works by setting the width of ``right'' to triple the width of
% its sister, label_3. (A bit of geometry shows why :-)
\def\qbalance{\qsetw{3\wd\treeboxtwo}}

% And a debugging/learning aid:
% Draw a (tight) frame around all parts of the tree.
% Note that these will override any content of the format hooks.
\def\qtreeshowframes{\let\qlabelhook=\frame \let\qleafhook=\frame}
% We can include this in \qtreeshowframes, but it's too much information
%  \def\qbranch@hook{\setbox\treeboxone \hbox{\frame{\box\treeboxone}}}

\def\qtreenoframes{\let\qlabelhook=\relax \let\qleafhook=\relax}
%  \let \qbranch@hook=\relax}


% Put a frame around the just-completed node.
% 
\def\qframesubtree{\setbox\treeboxone \hbox{\framebox{\box\treeboxone}}}

%
\newcommand{\poptree}{%
\ifnum\value{treecount}=0\typeout{QobiTeX warning---Tree stack underflow}\fi
\addtocounter{treecount}{-1}%
\setlength{\treeoffsettwo}{\treeoffsetthree}%
\setlength{\treeoffsetthree}{\treeoffsetfour}%
\setlength{\treeoffsetfour}{\treeoffsetfive}%
\setlength{\treeoffsetfive}{\treeoffsetsix}%
\setlength{\treeoffsetsix}{\treeoffsetseven}%
\setlength{\treeoffsetseven}{\treeoffseteight}%
\setlength{\treeoffseteight}{\treeoffsetnine}%
\setlength{\treeoffsetnine}{\treeoffsetten}%
\setlength{\treeoffsetten}{\treeoffseteleven}%
\setlength{\treeoffseteleven}{\treeoffsettwelve}%
\setlength{\treeoffsettwelve}{\treeoffsetthirteen}%
\setlength{\treeoffsetthirteen}{\treeoffsetfourteen}%
\setlength{\treeoffsetfourteen}{\treeoffsetfifteen}%
\setlength{\treeoffsetfifteen}{\treeoffsetsixteen}%
\setlength{\treeoffsetsixteen}{\treeoffsetseventeen}%
\setlength{\treeoffsetseventeen}{\treeoffseteighteen}%
\setlength{\treeoffseteighteen}{\treeoffsetnineteen}%
\setlength{\treeoffsetnineteen}{\treeoffsettwenty}%
\setlength{\treeshifttwo}{\treeshiftthree}%
\setlength{\treeshiftthree}{\treeshiftfour}%
\setlength{\treeshiftfour}{\treeshiftfive}%
\setlength{\treeshiftfive}{\treeshiftsix}%
\setlength{\treeshiftsix}{\treeshiftseven}%
\setlength{\treeshiftseven}{\treeshifteight}%
\setlength{\treeshifteight}{\treeshiftnine}%
\setlength{\treeshiftnine}{\treeshiftten}%
\setlength{\treeshiftten}{\treeshifteleven}%
\setlength{\treeshifteleven}{\treeshifttwelve}%
\setlength{\treeshifttwelve}{\treeshiftthirteen}%
\setlength{\treeshiftthirteen}{\treeshiftfourteen}%
\setlength{\treeshiftfourteen}{\treeshiftfifteen}%
\setlength{\treeshiftfifteen}{\treeshiftsixteen}%
\setlength{\treeshiftsixteen}{\treeshiftseventeen}%
\setlength{\treeshiftseventeen}{\treeshifteighteen}%
\setlength{\treeshifteighteen}{\treeshiftnineteen}%
\setlength{\treeshiftnineteen}{\treeshifttwenty}%
\setlength{\treewidthtwo}{\treewidththree}%
\setlength{\treewidththree}{\treewidthfour}%
\setlength{\treewidthfour}{\treewidthfive}%
\setlength{\treewidthfive}{\treewidthsix}%
\setlength{\treewidthsix}{\treewidthseven}%
\setlength{\treewidthseven}{\treewidtheight}%
\setlength{\treewidtheight}{\treewidthnine}%
\setlength{\treewidthnine}{\treewidthten}%
\setlength{\treewidthten}{\treewidtheleven}%
\setlength{\treewidtheleven}{\treewidthtwelve}%
\setlength{\treewidthtwelve}{\treewidththirteen}%
\setlength{\treewidththirteen}{\treewidthfourteen}%
\setlength{\treewidthfourteen}{\treewidthfifteen}%
\setlength{\treewidthfifteen}{\treewidthsixteen}%
\setlength{\treewidthsixteen}{\treewidthseventeen}%
\setlength{\treewidthseventeen}{\treewidtheighteen}%
\setlength{\treewidtheighteen}{\treewidthnineteen}%
\setlength{\treewidthnineteen}{\treewidthtwenty}%
\setbox\treeboxtwo \box\treeboxthree
\setbox\treeboxthree \box\treeboxfour
\setbox\treeboxfour \box\treeboxfive
\setbox\treeboxfive \box\treeboxsix
\setbox\treeboxsix \box\treeboxseven
\setbox\treeboxseven \box\treeboxeight
\setbox\treeboxeight \box\treeboxnine
\setbox\treeboxnine \box\treeboxten
\setbox\treeboxten \box\treeboxeleven
\setbox\treeboxeleven \box\treeboxtwelve
\setbox\treeboxtwelve \box\treeboxthirteen
\setbox\treeboxthirteen \box\treeboxfourteen
\setbox\treeboxfourteen \box\treeboxfifteen
\setbox\treeboxfifteen \box\treeboxsixteen
\setbox\treeboxsixteen \box\treeboxseventeen
\setbox\treeboxseventeen \box\treeboxeighteen
\setbox\treeboxeighteen \box\treeboxnineteen
\setbox\treeboxnineteen \box\treeboxtwenty }
%
\newcommand{\leaf}[1]{%
\ifnum\value{treecount}=20\typeout{QobiTeX warning---Tree stack overflow}\fi%
\addtocounter{treecount}{1}%
\setbox\treeboxtwenty \box\treeboxnineteen
\setbox\treeboxnineteen \box\treeboxeighteen
\setbox\treeboxeighteen \box\treeboxseventeen
\setbox\treeboxseventeen \box\treeboxsixteen
\setbox\treeboxsixteen \box\treeboxfifteen
\setbox\treeboxfifteen \box\treeboxfourteen
\setbox\treeboxfourteen \box\treeboxthirteen
\setbox\treeboxthirteen \box\treeboxtwelve
\setbox\treeboxtwelve \box\treeboxeleven
\setbox\treeboxeleven \box\treeboxten
\setbox\treeboxten \box\treeboxnine
\setbox\treeboxnine \box\treeboxeight
\setbox\treeboxeight \box\treeboxseven
\setbox\treeboxseven \box\treeboxsix
\setbox\treeboxsix \box\treeboxfive
\setbox\treeboxfive \box\treeboxfour
\setbox\treeboxfour \box\treeboxthree
\setbox\treeboxthree \box\treeboxtwo
\setbox\treeboxtwo \box\treeboxone
\sbox{\treeboxone}{\qshow@leaf{#1}}%
\sbox{\treeboxone}{\raisebox{-\ht\treeboxone}{\usebox{\treeboxone}}}%
\setlength{\treeoffsettwenty}{\treeoffsetnineteen}%
\setlength{\treeoffsetnineteen}{\treeoffseteighteen}%
\setlength{\treeoffseteighteen}{\treeoffsetseventeen}%
\setlength{\treeoffsetseventeen}{\treeoffsetsixteen}%
\setlength{\treeoffsetsixteen}{\treeoffsetfifteen}%
\setlength{\treeoffsetfifteen}{\treeoffsetfourteen}%
\setlength{\treeoffsetfourteen}{\treeoffsetthirteen}%
\setlength{\treeoffsetthirteen}{\treeoffsettwelve}%
\setlength{\treeoffsettwelve}{\treeoffseteleven}%
\setlength{\treeoffseteleven}{\treeoffsetten}%
\setlength{\treeoffsetten}{\treeoffsetnine}%
\setlength{\treeoffsetnine}{\treeoffseteight}%
\setlength{\treeoffseteight}{\treeoffsetseven}%
\setlength{\treeoffsetseven}{\treeoffsetsix}%
\setlength{\treeoffsetsix}{\treeoffsetfive}%
\setlength{\treeoffsetfive}{\treeoffsetfour}%
\setlength{\treeoffsetfour}{\treeoffsetthree}%
\setlength{\treeoffsetthree}{\treeoffsettwo}%
\setlength{\treeoffsettwo}{\treeoffsetone}%
\setlength{\treeoffsetone}{0.5\wd\treeboxone}%
\setlength{\treeshifttwenty}{\treeshiftnineteen}%
\setlength{\treeshiftnineteen}{\treeshifteighteen}%
\setlength{\treeshifteighteen}{\treeshiftseventeen}%
\setlength{\treeshiftseventeen}{\treeshiftsixteen}%
\setlength{\treeshiftsixteen}{\treeshiftfifteen}%
\setlength{\treeshiftfifteen}{\treeshiftfourteen}%
\setlength{\treeshiftfourteen}{\treeshiftthirteen}%
\setlength{\treeshiftthirteen}{\treeshifttwelve}%
\setlength{\treeshifttwelve}{\treeshifteleven}%
\setlength{\treeshifteleven}{\treeshiftten}%
\setlength{\treeshiftten}{\treeshiftnine}%
\setlength{\treeshiftnine}{\treeshifteight}%
\setlength{\treeshifteight}{\treeshiftseven}%
\setlength{\treeshiftseven}{\treeshiftsix}%
\setlength{\treeshiftsix}{\treeshiftfive}%
\setlength{\treeshiftfive}{\treeshiftfour}%
\setlength{\treeshiftfour}{\treeshiftthree}%
\setlength{\treeshiftthree}{\treeshifttwo}%
\setlength{\treeshifttwo}{\treeshiftone}%
\setlength{\treeshiftone}{0pt}%
\setlength{\treewidthtwenty}{\treewidthnineteen}%
\setlength{\treewidthnineteen}{\treewidtheighteen}%
\setlength{\treewidtheighteen}{\treewidthseventeen}%
\setlength{\treewidthseventeen}{\treewidthsixteen}%
\setlength{\treewidthsixteen}{\treewidthfifteen}%
\setlength{\treewidthfifteen}{\treewidthfourteen}%
\setlength{\treewidthfourteen}{\treewidththirteen}%
\setlength{\treewidththirteen}{\treewidthtwelve}%
\setlength{\treewidthtwelve}{\treewidtheleven}%
\setlength{\treewidtheleven}{\treewidthten}%
\setlength{\treewidthten}{\treewidthnine}%
\setlength{\treewidthnine}{\treewidtheight}%
\setlength{\treewidtheight}{\treewidthseven}%
\setlength{\treewidthseven}{\treewidthsix}%
\setlength{\treewidthsix}{\treewidthfive}%
\setlength{\treewidthfive}{\treewidthfour}%
\setlength{\treewidthfour}{\treewidththree}%
\setlength{\treewidththree}{\treewidthtwo}%
\setlength{\treewidthtwo}{\treewidthone}%
\setlength{\treewidthone}{\wd\treeboxone}}
%
\newcommand{\branch}[2]{%
\setcounter{branchcount}{#1}%
% 0-branching
\ifnum\value{branchcount}=0\leaf{#2}\else% added 2002/04/19, RIG
% Branch label (all branch counts)
\sbox{\parentbox}{\qshow@text{\qlabelhook}{#2}}%
% 1-branching
\ifnum\value{branchcount}=1
\setlength{\parentoffset}{\treeoffsetone}%
\addtolength{\parentoffset}{-0.5\wd\parentbox}%
\setlength{\daughteroffset}{0in}%
\ifdim\parentoffset<0in%
\setlength{\daughteroffset}{-\parentoffset}%
\setlength{\parentoffset}{0in}\fi%
\setlength{\parentwidth}{\parentoffset}%
\addtolength{\parentwidth}{\wd\parentbox}%
\setlength{\treeoffset}{\daughteroffset}%
\addtolength{\treeoffset}{\treeoffsetone}%
\setlength{\treewidth}{\wd\treeboxone}%
\addtolength{\treewidth}{\daughteroffset}%
\ifdim\treewidth<\parentwidth\setlength{\treewidth}{\parentwidth}\fi%
\sbox{\treebox}{\begin{minipage}{\treewidth}%
\begin{flushleft}%
\showparentbox
{\setlength{\unitlength}{\qtreeunaryht}%
\hspace*{\treeoffset}\qdraw@branches{1}}\\  
\vspace{-\baselineskip}%
\hspace*{\daughteroffset}%
\raisebox{-\ht\treeboxone}{\usebox{\treeboxone}}%
\end{flushleft}%
\end{minipage}}%
\setlength{\treeoffsetone}{\parentoffset}%
\addtolength{\treeoffsetone}{0.5\wd\parentbox}%
\setlength{\treeshiftone}{0pt}%
\setlength{\treewidthone}{\treewidth}%
\sbox{\treeboxone}{\usebox{\treebox}}%
% 2-branching
\else\ifnum\value{branchcount}=2
\setlength{\branchwidthone}{\treewidthtwo}%
\addtolength{\branchwidthone}{\treeoffsetone}%
\addtolength{\branchwidthone}{-\treeshiftone}%
\addtolength{\branchwidthone}{-\treeoffsettwo}%
\setlength{\branchwidth}{\branchwidthone}%
\setlength{\daughteroffsetone}{\branchwidth}%
\addtolength{\daughteroffsetone}{-\branchwidthone}%
\addtolength{\daughteroffsetone}{-\treeshiftone}%
\setlength{\parentoffset}{-0.5\wd\parentbox}%
\addtolength{\parentoffset}{\treeoffsettwo}%
\addtolength{\parentoffset}{0.5\branchwidth}%
\setlength{\daughteroffset}{0in}%
\ifdim\parentoffset<0in%
\setlength{\daughteroffset}{-\parentoffset}%
\setlength{\parentoffset}{0in}\fi%
\setlength{\parentwidth}{\parentoffset}%
\addtolength{\parentwidth}{\wd\parentbox}%
\setlength{\treeoffset}{\daughteroffset}%
\addtolength{\treeoffset}{\treeoffsettwo}%
\setlength{\treewidth}{\wd\treeboxone}%
\addtolength{\treewidth}{\daughteroffsetone}%
\addtolength{\treewidth}{\treewidthtwo}%
\addtolength{\treewidth}{\daughteroffset}%
\ifdim\treewidth<\parentwidth\setlength{\treewidth}{\parentwidth}\fi%
\sbox{\treebox}{\begin{minipage}{\treewidth}%
\begin{flushleft}%
\showparentbox %\frame
{\setlength{\unitlength}{0.5\branchwidth}%
\hspace*{\treeoffset}\qdraw@branches{2}}\\
\vspace{-\baselineskip}%
\hspace*{\daughteroffset}%
\makebox[\treewidthtwo][l]%
{\raisebox{-\ht\treeboxtwo}{\usebox{\treeboxtwo}}}%
\hspace*{\daughteroffsetone}%
\raisebox{-\ht\treeboxone}{\usebox{\treeboxone}}%
\end{flushleft}%
\end{minipage}}%
\setlength{\treeoffsetone}{\parentoffset}%
\addtolength{\treeoffsetone}{0.5\wd\parentbox}%
\setlength{\treeshiftone}{0pt}%
\setlength{\treewidthone}{\treewidth}%
\sbox{\treeboxone}{\usebox{\treebox}}\poptree%
% 3-branching
\else\ifnum\value{branchcount}=3
\setlength{\branchwidthone}{\treewidthtwo}%
\addtolength{\branchwidthone}{\treeoffsetone}%
\addtolength{\branchwidthone}{-\treeshiftone}%
\addtolength{\branchwidthone}{-\treeoffsettwo}%
\setlength{\branchwidthtwo}{\treewidththree}%
\addtolength{\branchwidthtwo}{\treeoffsettwo}%
\addtolength{\branchwidthtwo}{-\treeshifttwo}%
\addtolength{\branchwidthtwo}{-\treeoffsetthree}%
\setlength{\branchwidth}{\branchwidthone}%
\ifdim\branchwidthtwo>\branchwidth%
\setlength{\branchwidth}{\branchwidthtwo}\fi%
\setlength{\daughteroffsetone}{\branchwidth}%
\addtolength{\daughteroffsetone}{-\branchwidthone}%
\addtolength{\daughteroffsetone}{-\treeshiftone}%
\setlength{\daughteroffsettwo}{\branchwidth}%
\addtolength{\daughteroffsettwo}{-\branchwidthtwo}%
\addtolength{\daughteroffsettwo}{-\treeshifttwo}%
\setlength{\parentoffset}{-0.5\wd\parentbox}%
\addtolength{\parentoffset}{\treeoffsetthree}%
\addtolength{\parentoffset}{\branchwidth}%
\setlength{\daughteroffset}{0in}%
\ifdim\parentoffset<0in%
\setlength{\daughteroffset}{-\parentoffset}%
\setlength{\parentoffset}{0in}\fi%
\setlength{\parentwidth}{\parentoffset}%
\addtolength{\parentwidth}{\wd\parentbox}%
\setlength{\treeoffset}{\daughteroffset}%
\addtolength{\treeoffset}{\treeoffsetthree}%
\setlength{\treewidth}{\wd\treeboxone}%
\addtolength{\treewidth}{\daughteroffsetone}%
\addtolength{\treewidth}{\treewidthtwo}%
\addtolength{\treewidth}{\daughteroffsettwo}%
\addtolength{\treewidth}{\treewidththree}%
\addtolength{\treewidth}{\daughteroffset}%
\ifdim\treewidth<\parentwidth\setlength{\treewidth}{\parentwidth}\fi%
\sbox{\treebox}{\begin{minipage}{\treewidth}%
\begin{flushleft}%
\showparentbox
{\setlength{\unitlength}{0.5\branchwidth}%
\hspace*{\treeoffset}\qdraw@branches{3}}\\
\vspace{-\baselineskip}%
\hspace*{\daughteroffset}%
\makebox[\treewidththree][l]%
{\raisebox{-\ht\treeboxthree}{\usebox{\treeboxthree}}}%
\hspace*{\daughteroffsettwo}%
\makebox[\treewidthtwo][l]%
{\raisebox{-\ht\treeboxtwo}{\usebox{\treeboxtwo}}}%
\hspace*{\daughteroffsetone}%
\raisebox{-\ht\treeboxone}{\usebox{\treeboxone}}%
\end{flushleft}%
\end{minipage}}%
\setlength{\treeoffsetone}{\parentoffset}%
\addtolength{\treeoffsetone}{0.5\wd\parentbox}%
\setlength{\treeshiftone}{0pt}%
\setlength{\treewidthone}{\treewidth}%
\sbox{\treeboxone}{\usebox{\treebox}}\poptree\poptree%
% 4-branching
\else\ifnum\value{branchcount}=4
\setlength{\branchwidthone}{\treewidthtwo}%
\addtolength{\branchwidthone}{\treeoffsetone}%
\addtolength{\branchwidthone}{-\treeshiftone}%
\addtolength{\branchwidthone}{-\treeoffsettwo}%
\setlength{\branchwidthtwo}{\treewidththree}%
\addtolength{\branchwidthtwo}{\treeoffsettwo}%
\addtolength{\branchwidthtwo}{-\treeshifttwo}%
\addtolength{\branchwidthtwo}{-\treeoffsetthree}%
\setlength{\branchwidththree}{\treewidthfour}%
\addtolength{\branchwidththree}{\treeoffsetthree}%
\addtolength{\branchwidththree}{-\treeshiftthree}%
\addtolength{\branchwidththree}{-\treeoffsetfour}%
\setlength{\branchwidth}{\branchwidthone}%
\ifdim\branchwidthtwo>\branchwidth%
\setlength{\branchwidth}{\branchwidthtwo}\fi%
\ifdim\branchwidththree>\branchwidth%
\setlength{\branchwidth}{\branchwidththree}\fi%
\setlength{\daughteroffsetone}{\branchwidth}%
\addtolength{\daughteroffsetone}{-\branchwidthone}%
\addtolength{\daughteroffsetone}{-\treeshiftone}%
\setlength{\daughteroffsettwo}{\branchwidth}%
\addtolength{\daughteroffsettwo}{-\branchwidthtwo}%
\addtolength{\daughteroffsettwo}{-\treeshifttwo}%
\setlength{\daughteroffsetthree}{\branchwidth}%
\addtolength{\daughteroffsetthree}{-\branchwidththree}%
\addtolength{\daughteroffsetthree}{-\treeshiftthree}%
\setlength{\parentoffset}{-0.5\wd\parentbox}%
\addtolength{\parentoffset}{\treeoffsetfour}%
\addtolength{\parentoffset}{1.5\branchwidth}%
\setlength{\daughteroffset}{0in}%
\ifdim\parentoffset<0in%
\setlength{\daughteroffset}{-\parentoffset}%
\setlength{\parentoffset}{0in}\fi%
\setlength{\parentwidth}{\parentoffset}%
\addtolength{\parentwidth}{\wd\parentbox}%
\setlength{\treeoffset}{\daughteroffset}%
\addtolength{\treeoffset}{\treeoffsetfour}%
\setlength{\treewidth}{\wd\treeboxone}%
\addtolength{\treewidth}{\daughteroffsetone}%
\addtolength{\treewidth}{\treewidthtwo}%
\addtolength{\treewidth}{\daughteroffsettwo}%
\addtolength{\treewidth}{\treewidththree}%
\addtolength{\treewidth}{\daughteroffsetthree}%
\addtolength{\treewidth}{\treewidthfour}%
\addtolength{\treewidth}{\daughteroffset}%
\ifdim\treewidth<\parentwidth\setlength{\treewidth}{\parentwidth}\fi%
\sbox{\treebox}{\begin{minipage}{\treewidth}%
\begin{flushleft}%
\showparentbox
{\setlength{\unitlength}{0.5\branchwidth}%
\hspace*{\treeoffset}\qdraw@branches{4}}\\
\vspace{-\baselineskip}%
\hspace*{\daughteroffset}%
\makebox[\treewidthfour][l]%
{\raisebox{-\ht\treeboxfour}{\usebox{\treeboxfour}}}%
\hspace*{\daughteroffsetthree}%
\makebox[\treewidththree][l]%
{\raisebox{-\ht\treeboxthree}{\usebox{\treeboxthree}}}%
\hspace*{\daughteroffsettwo}%
\makebox[\treewidthtwo][l]%
{\raisebox{-\ht\treeboxtwo}{\usebox{\treeboxtwo}}}%
\hspace*{\daughteroffsetone}%
\raisebox{-\ht\treeboxone}{\usebox{\treeboxone}}%
\end{flushleft}%
\end{minipage}}%
\setlength{\treeoffsetone}{\parentoffset}%
\addtolength{\treeoffsetone}{0.5\wd\parentbox}%
\setlength{\treeshiftone}{0pt}%
\setlength{\treewidthone}{\treewidth}%
\sbox{\treeboxone}{\usebox{\treebox}}\poptree\poptree\poptree%
% 5-branching
\else\ifnum\value{branchcount}=5
\setlength{\branchwidthone}{\treewidthtwo}%
\addtolength{\branchwidthone}{\treeoffsetone}%
\addtolength{\branchwidthone}{-\treeshiftone}%
\addtolength{\branchwidthone}{-\treeoffsettwo}%
\setlength{\branchwidthtwo}{\treewidththree}%
\addtolength{\branchwidthtwo}{\treeoffsettwo}%
\addtolength{\branchwidthtwo}{-\treeshifttwo}%
\addtolength{\branchwidthtwo}{-\treeoffsetthree}%
\setlength{\branchwidththree}{\treewidthfour}%
\addtolength{\branchwidththree}{\treeoffsetthree}%
\addtolength{\branchwidththree}{-\treeshiftthree}%
\addtolength{\branchwidththree}{-\treeoffsetfour}%
\setlength{\branchwidthfour}{\treewidthfive}%
\addtolength{\branchwidthfour}{\treeoffsetfour}%
\addtolength{\branchwidthfour}{-\treeshiftfour}%
\addtolength{\branchwidthfour}{-\treeoffsetfive}%
\setlength{\branchwidth}{\branchwidthone}%
\ifdim\branchwidthtwo>\branchwidth%
\setlength{\branchwidth}{\branchwidthtwo}\fi%
\ifdim\branchwidththree>\branchwidth%
\setlength{\branchwidth}{\branchwidththree}\fi%
\ifdim\branchwidthfour>\branchwidth%
\setlength{\branchwidth}{\branchwidthfour}\fi%
\setlength{\daughteroffsetone}{\branchwidth}%
\addtolength{\daughteroffsetone}{-\branchwidthone}%
\addtolength{\daughteroffsetone}{-\treeshiftone}%
\setlength{\daughteroffsettwo}{\branchwidth}%
\addtolength{\daughteroffsettwo}{-\branchwidthtwo}%
\addtolength{\daughteroffsettwo}{-\treeshifttwo}%
\setlength{\daughteroffsetthree}{\branchwidth}%
\addtolength{\daughteroffsetthree}{-\branchwidththree}%
\addtolength{\daughteroffsetthree}{-\treeshiftthree}%
\setlength{\daughteroffsetfour}{\branchwidth}%
\addtolength{\daughteroffsetfour}{-\branchwidthfour}%
\addtolength{\daughteroffsetfour}{-\treeshiftfour}%
\setlength{\parentoffset}{-0.5\wd\parentbox}%
\addtolength{\parentoffset}{\treeoffsetfive}%
\addtolength{\parentoffset}{2\branchwidth}%
\setlength{\daughteroffset}{0in}%
\ifdim\parentoffset<0in%
\setlength{\daughteroffset}{-\parentoffset}%
\setlength{\parentoffset}{0in}\fi%
\setlength{\parentwidth}{\parentoffset}%
\addtolength{\parentwidth}{\wd\parentbox}%
\setlength{\treeoffset}{\daughteroffset}%
\addtolength{\treeoffset}{\treeoffsetfive}%
\setlength{\treewidth}{\wd\treeboxone}%
\addtolength{\treewidth}{\daughteroffsetone}%
\addtolength{\treewidth}{\treewidthtwo}%
\addtolength{\treewidth}{\daughteroffsettwo}%
\addtolength{\treewidth}{\treewidththree}%
\addtolength{\treewidth}{\daughteroffsetthree}%
\addtolength{\treewidth}{\treewidthfour}%
\addtolength{\treewidth}{\daughteroffsetfour}%
\addtolength{\treewidth}{\treewidthfive}%
\addtolength{\treewidth}{\daughteroffset}%
\ifdim\treewidth<\parentwidth\setlength{\treewidth}{\parentwidth}\fi%
\sbox{\treebox}{\begin{minipage}{\treewidth}%
\begin{flushleft}%
\showparentbox
{\setlength{\unitlength}{0.5\branchwidth}%
\hspace*{\treeoffset}\qdraw@branches{5}}\\
\vspace{-\baselineskip}%
\hspace*{\daughteroffset}%
\makebox[\treewidthfive][l]%
{\raisebox{-\ht\treeboxfive}{\usebox{\treeboxfive}}}%
\hspace*{\daughteroffsetfour}%
\makebox[\treewidthfour][l]%
{\raisebox{-\ht\treeboxfour}{\usebox{\treeboxfour}}}%
\hspace*{\daughteroffsetthree}%
\makebox[\treewidththree][l]%
{\raisebox{-\ht\treeboxthree}{\usebox{\treeboxthree}}}%
\hspace*{\daughteroffsettwo}%
\makebox[\treewidthtwo][l]%
{\raisebox{-\ht\treeboxtwo}{\usebox{\treeboxtwo}}}%
\hspace*{\daughteroffsetone}%
\raisebox{-\ht\treeboxone}{\usebox{\treeboxone}}%
\end{flushleft}%
\end{minipage}}%
\setlength{\treeoffsetone}{\parentoffset}%
\addtolength{\treeoffsetone}{0.5\wd\parentbox}%
\setlength{\treeshiftone}{0pt}%
\setlength{\treewidthone}{\treewidth}%
\sbox{\treeboxone}{\usebox{\treebox}}\poptree\poptree\poptree\poptree%
\else\typeout{QobiTeX warning--- Can't handle #1 branching}\fi\fi\fi\fi\fi\fi}
%
\newcommand{\faketreewidth}[1]{%
  \sbox{\parentbox}{\qshow@text{\relax}{#1}}%
  \setlength{\treewidthone}{0.5\wd\parentbox}%
  \addtolength{\treewidthone}{\treeoffsetone}%
  \setlength{\treeshiftone}{\treeoffsetone}%
  \addtolength{\treeshiftone}{-0.5\wd\parentbox}}
%

% set-tree-width: like \faketreewidth, but takes a dimension argument
% Use: \qsetw{1in} etc.
\newcommand{\qsetw}[1]{%
  \treewidthone = #1
  \treewidthone = 0.5\treewidthone
  \treeshiftone = -\treewidthone
  \advance\treewidthone by \treeoffsetone
  \advance\treeshiftone by \treeoffsetone}

% Fix up alignment of the tree and print out.
% \Tree has already taken care of centering, etc.
%
\newcommand{\qobitree}{%
  % Cancel padding inserted by \qshow@text
  \setbox\treeboxone \hbox{%
      \hskip-\qtreepadding \box\treeboxone \hskip-\qtreepadding}%
  % Lower box, to top-align with baseline
  \@tempdima=\ht\treeboxone
  \advance\@tempdima by -\ht\strutbox
  \leavevmode \raise-\@tempdima \box\treeboxone
  % Now clean up
  \setlength{\treeoffsetone}{\treeoffsettwo}%
  \setbox\treeboxone \box\treeboxtwo
  \poptree }



%%%%%%%%%%%%%%%%%%%%%%%%%%%%%%%%%%%%%%%%%%%%%%%%%%%%%%%%%%%%%%%%%%%%%%%%%

%%%%%%%%%%%%%%%%%%%%%%%%%%%%%%%%%%%%%%%%%%%%%%%%%%%%%%%%%%%%%%%%%%%%%%%%%

%%%%%%%%%%%%%%%%%%%%%%%%%%%%%%%%%%%%%%%%%%%%%%%%%%%%%%%%%%%%%%%%%%%%%%%%%

% Front end for qobitree.  Reads a tree in bracketed notation and
% generates commands to build the specified tree.
% Tokens are space-delimited; brackets {} may be used to alter grouping.
%
% Alexis Dimitriadis (alexis@babel.ling.upenn.edu), Dec 6, 1993.
%
% Sample input:
% \Tree [ [ John ].NP [ has [ [ seen ].V [ {the book} ].NP ].T ].IP ].S
% To insert extra material, precede token with an exclamation mark, e.g.
%       ... ].V !\faketreewidth{VPP} ...

%%%%%%%%%%%%%%%%%%%%%%%%%%%%%%%%%%%%%%%%%%%%%%%%%%%%%%%%%%%%%%%%%%%%%%%%%
% Very messy stack macros, to make up for lack of nested environments in
% qobitree. 
%
% \Spush\Stack{X}   Push X on stack \Stack.
% \Spop\Stack       Pop top element off \Stack, leave in the input stream.
% \Spopd\Stack\to\v Define \v to be the top element of \Stack, pop it.
%

% This will trigger a LaTeX diagnostic if we pop too far.
\def\qbstack{<Stack Underflow>}
\def\qnstack{<Stack Underflow>}


\newtoks\qta \newtoks\qtb
% Expand #2 and push on stack #1.  Things already pushed are not
% reexpanded.
\long\def\Spush#1#2{\qta=\expandafter{#1}%
  \edef#1{{#2}\noexpand\@@STP{\the\qta}}}

% Push #2 on stack #1 unexpanded.
%
\long\def\SpushU#1#2{\qta=\expandafter{#1}\qtb={#2}%
  \edef#1{{\the\qtb}\noexpand\@@STP{\the\qta}}}

\def\@@STP{\def\@@StpV}

% Pop the top element of stack #1, leaving in the input stream.
%
\def\Spop#1{#1\let#1=\@@StpV}

% Pop the top element of stack #1, expand it, and define #2 to it.
% 
\def\SpopD#1#2{\expandafter\qta#1\let#1=\@@StpV\edef#2{\the\qta}}

%%%%%%%%%%%%%%%%%%%%%%%%%%%%%%%%%%%%%%%%%%%%%%%%%%%%%%%%%%%%%%%%%%%%%%%%%
%
% This is the front end proper.  Everything else is just sugar.
% (the stack macros are necessary).
%
\newcount\nbranches

\def\Tree{\bgroup
  \@ifundefined{qtreebugfixhook}{}\qtreebugfixhook % retained for compatibility
  \automath \qtreeprimes@internal
  \qtreeinithook
  \ifqtreecenter\hskip 0pt plus 1fil\else\leavevmode\fi 
  \nbranches=0\relax \q@recurse }

\def\endTree{\qobitree \qtreefinalhook \egroup}

\def\q@recurse[{\@ifnextchar.{\qq@recurse}{\qq@recurse. }}
\def\qq@recurse.#1 {\SpushU\qnstack{#1}%
  \Spush\qbstack{\number\nbranches}\nbranches=0\relax \q@lookfornodes } 

% \doanode should more properly be called \doaleaf.
% \@qseenode is the first token of our argument: handle specially if a roof.
% The space in \leaf is for terminating the argument of \qroof
\def\doanode#1 {\advance\nbranches by1 
    \ifx\@qseenode\qroof \leaf{#1 }%
    \else \leaf{#1}\fi \q@lookfornodes} 

\def\q@lookfornodes{\@ifnextchar]{\closeoff}{%
  \@ifnextchar[{\advance\nbranches by1\relax \q@recurse}{%
      \@ifnextchar!{\pushliteral}{\futurelet\@qseenode\doanode}}}}

\def\pushliteral!#1 {#1\relax \q@lookfornodes}

\def\closeoff]{\@ifnextchar.{\expandafter\@closeoff}{\expandafter\@closeoff. }}
% If either label is missing, use the other.
% Otherwise, disallow mismatched labels
% Push and pop right argument to make sure the tokens are in same state...
\def\@closeoff.#1 {%
  \def\rarg{#1}%
  \SpopD\qnstack\larg
  \ifx\larg\@empty \let\larg=\rarg 
  \else \ifx\rarg\@empty \let\rarg=\larg \fi\fi
  \ifx\larg\rarg \else 
    \errmessage{MISMATCHED LABELS, [.\larg\ ... ].\rarg}%
  \fi 
  \@@closeoff\larg }

\def\@@closeoff#1{\branch{\number\nbranches}{#1}%
  \SpopD\qbstack\x\nbranches=\x \relax
% \showthe\nbranches
  \qbranch@hook
  \ifnum\nbranches>0 \expandafter\q@lookfornodes
  \else \expandafter\endTree \fi}

%%%%%%%%%%%%%%%%%%%%%%%%%%%%%%%%%%%%%%%%%%%%%%%%%%%%%%%%%%%%%%%%%%%%%%%%%

% and another odd convenience:
%
% Make _, ^ go into math mode automatically in the scope of \automath
%
{ % Temporarily change catcodes
  \catcode`\_=\active
  \catcode`\^=\active

  \global\def\automath{%
    \catcode`\_=\active
    \catcode`\^=\active
    \def_##1{\@ifnextchar^{\automath@two_{##1}}{\ensuremath{\sb{##1}}}}%
    \def^##1{\@ifnextchar_{\automath@two^{##1}}{\ensuremath{\sp{##1}}}}}
}
\def\automath@two#1#2#3#4{\ensuremath{#1{#2}\relax #3{#4}}}
% Restore default catcodes for ^, _
\def\noautomath{\catcode`\_=8 \catcode`\^=7 }


% Let \0, \1, \2 produce ^0, $'$, $''$
% The \rlap results in better centering of the label (ignoring the
%  superscript)
\def\qtreeprimes@internal{%
  \def\0{\ifmmode ^0\else \rlap{$^0$}\fi}%
  \def\1{\ifmmode '\else \rlap{$'$}\fi}%
  \def\2{\ifmmode ''\else \rlap{$''$}\fi}%
}

% Same commands, but without the \rlap feature
\def\qtreeprimes{%
  \def\0{\ensuremath{^0}}%
  \def\1{\ensuremath{'}}%
  \def\2{\ensuremath{''}}}

%%%%%%%%%%%%%%%%%%%%%%%%%%%%%%%%%%%%%%%%%%%%%%%%%%%%%%%%%%%%%%%%%%%%%%%%%

% qroof: Build a triangular ``roof'' with label #2 and contents (under the
% roof) #1.  The width of the roof is computed automatically.  The contents
% may contain line breaks (\\).
%

% The slope of the roof built by qroof (may be changed anywhere).
\newcount\qroofx
\newcount\qroofy
\qroofx=3 \qroofy=1

\newbox\@qrscratchbox

% User command for requesting a roof
%
\def\qroof#1.#2 {{%
  % padding under the "eaves" of the roof
  \setbox\@qrscratchbox = \hbox{\let\qtreepadding=\qroofpadding
    \qshow@text[l]{\qleafhook}{#1}}%
    % we don't pad here, since the entire roof gets padded as a leaf.
  \def\qtreepadding{0pt}%
  \begin{tabular}{@{}c@{}}
      \setbox\@tempboxa = \hbox{\qshow@text{\qlabelhook}{#2}}%
          \ifdim\wd\@tempboxa>0pt \box\@tempboxa \\ \fi
      \unitlength=\wd\@qrscratchbox \qdraw@roof \\[-0.6ex]
      \box\@qrscratchbox
\end{tabular}}}


%%%%%%%%%%%%%%%%%%%%%%%%%%%%%%%%%%%%%%%%%%%%%%%%%%%%%%%%%%%%%%%%%%%%%%%%%
%
%                          DRAWING UTILITIES

% The drawing routines may be redefined. The new drawing configuration
% must provide the commands:
%
% \qdraw@branches{N}   Draw N branches, each \unitlength wide
% \qdraw@roof          Draw a roof, \unitlength wide
%
% (\unitlength is set to the proper value before these are called)


%%%%%%%%%%%%%%%%%%%%%%%%%%%%%%%%%%%%%%%%%%%%%%%%%%%%%%%%%%%%%%%%%
% The original, ``picture'' driver: Uses the native {picture} environment.
% It is expected that it will be augmented (originally with eepic, but now
% with pict2e.sty)

% Draws a roof with slope \qroofy/\qroofx, and width equal to the current
% width of \unitlength (which, during the actual drawing, is set to a
% suitable value
% Because we must use integer arithmetic, we divide \unitlength into
% 2\qroofx units for calculations
     
\def\qdraw@roof{{%  
  \def\wwd{\count1}%
  \wwd=\qroofx\relax
  \multiply\wwd by 2
  \divide\unitlength by \wwd
  \begin{picture}(\wwd,\qroofy)
    \put(0, 0){\line(1,0){\wwd}}
    \put(0, 0){\line(\qroofx, \qroofy){\qroofx}}
    \put(\wwd, 0){\line(-\qroofx, \qroofy){\qroofx}}
  \end{picture}}}

% Draw one of the branching structures we need, using the standard {picture}
% environment (with or without eepic extensions).
% The parameter \unitlength is already set appropriately
%
\def\qdraw@branches#1{\ifcase#1\relax % Zero case is unused
  \or  % One-branching
  \begin{picture}(0,1)
    \put(0,0){\line(0,1){1}}
  \end{picture}%
  \or % Two-branching
  \begin{picture}(2,0.5)
    \put(0,0){\line(2,1){1}}
    \put(2,0){\line(-2,1){1}}
  \end{picture}%
  \or % Three-branching
  \begin{picture}(4,1)
    \put(0,0){\line(2,1){2}}
    \put(2,0){\line(0,1){1}}
    \put(4,0){\line(-2,1){2}}
  \end{picture}%
  \or % Four-branching
  \begin{picture}(6,1)
    \put(0,0){\line(3,1){3}}
    \put(2,0){\line(1,1){1}}
    \put(4,0){\line(-1,1){1}}
    \put(6,0){\line(-3,1){3}}
  \end{picture}%
  \or % Five-branching
  \begin{picture}(8,1)
    \put(0,0){\line(4,1){4}}
    \put(2,0){\line(2,1){2}}
    \put(4,0){\line(0,1){1}}
    \put(6,0){\line(-2,1){2}}
    \put(8,0){\line(-4,1){4}}
  \end{picture}%
  \else\typeout{Qtree error--- Can't handle #1-way branching}
  \fi}

% END OF PICTURE driver definitions

